% -*- TeX -*- -*- RU -*-
\documentclass[russian,english]{book}

\usepackage{mathtext}
\usepackage[OT2]{fontenc}

%to run lh fonts only (default font if wnr10 instead of wncyr10)
\usepackage{lh-OT2}

%var1: lh additions for bold concrete fonts
\usepackage{lh-OT2xccr}
%var2: the package below could be covered with fd-files
%\usepackage{lh-OT2ccr}

\usepackage{babel}
%was needed for one version of russianb
%\languageattribute{russian}{mdq}

%do not create/modify aux-like files
%\nofiles

\addto\extrasrussian{\fontencoding{OT2}\selectfont}
\addto\extrasenglish{\fontencoding{OT1}\selectfont}

\begin{document}
\pagestyle{empty}
\selectlanguage{russian}

\centerline{*\quad*\quad*}
\begin{verse}%
Da! Teperp1 resheno. Bez vozvrata\\
YA pokinul rodnye polya.\\
Uzh ne budut listvoyu krylatoi0\\
Nado mnoyu zvenetp1 topolya.
\bigskip

Nizkii0 dom bez menya ssutulit{}sya,\\
Staryi0 pe0s moi0 davno izdoh.\\
Na moskovskih izognutyh ulicah\\
Umeretp1, znatp1, sudil mne Bog.
\bigskip

YA lyublyu e1tot gorod vyazevyi0,\\
Pustp1 obryuzg on i pustp1 odryah.\\
Zolotaya dremotnaya Aziya\\
Opochila na kupolah.
\bigskip

A kogda nochp1yu svetit mesyac.\\
Kogda svetit\ldots{} che0rt znaet kak!\\
YA idu, golovoyu svesyasp1,\\
Pereulkom v znakomyi0 kabak.
\bigskip

SHum i gam v e1tom logove zhutkom,\\
No vsyu nochp1 naprole0t, do zari,\\
YA chitayu stihi prostitutkam\\
I s banditami zharyu spirt.
\bigskip

Serdce bp1e0t{}sya vse chashche i chashche,\\
I uzh ya govoryu nevpopad:\\
"<YA takoi0 zhe, kak vy propashchii0,\\
Mne teperp1 ne ui0ti nazad">.
\bigskip

Nizkii0 dom bez menya ssutulit{}sya,\\
Staryi0 pe0s moi0 davno izdoh.\\
Na moskovskih izognutyh ulicah\\
Umeretp1, znatp1, sudil mne Bog.
\bigskip

\rightline{\small\slshape S.\,Esenin, 1922}
\end{verse}

\newpage
\fontfamily{ccr}\selectfont

\centerline{\bfseries SOBAKE KACHALOVA}
\begin{verse}%
Dai0, Dzhim, na schastp1e lapu mne,\\
Takuyu lapu ne vidal ya srodu.\\
Davai0 s toboi0 polaem pri lune\\
Na tihuyu besshumnuyu pogodu.\\
Dai0, Dzhim, na schastp1e lapu mne.
\bigskip

Pozhalui0sta, golubchik, ne lizhisp1.\\
Poi0mi so mnoi0 hotp1 samoe prostoe.\\
Vedp1 ty ne znaeshp1, chto takoe zhiznp1,\\
Ne znaeshp1 ty, chto zhitp1 na svete stoit.
\bigskip

Hozyain tvoi0 i mil i znamenit\\
I u nego gostei0 byvaet v dome mnogo,\\
I kazhdyi0, ulybayasp1, norovit\\
Tebya po shersti barhatnoi0 potrogatp1.
\bigskip

Ty po-sobachp1i dp1yavolp1ski krasiv,\\
S takoyu miloyu doverchivoi0 priyatcei0.\\
I, nikogo ni kapli ne sprosiv,\\
Kak pp1yanyi0 drug, ty lezeshp1 celovatp1sya.
\bigskip

Moi0 milyi0 Dzhim, sredi tvoih gostei0\\
Tak mnogo vsyakih i nevsyakih bylo.\\
No ta, chto vseh bezmolvnei0 i grustnei0,\\
Syuda sluchai0no vdrug ne zahodila?
\bigskip

Ona pride0t, dayu tebe poruku.\\
I bez menya, v ee0 ustavyasp1 vzglyad,\\
Ty za menya lizni ei0 nezhno ruku\\
Za vse0, v che0m byl i ne byl vinovat.
\bigskip

\rightline{\small\slshape S.\,Esenin, 1925}
\end{verse}

\newpage
\tolerance2000

\fontfamily{cmbr}\selectfont

\ldots

\cdash--* Sudarynya, \cdash--- govoryu ya, \cdash--- ostorozhnee povorachivai0te mladenca, ne zabudp1te, chto on rozhde0n ranee sroka. Smertp1 e1togo mladenca oznachala by tyazhelei0shuyu utratu dlya vashei0 strany!

\cdash--* Moi0 bog! Gospozha Poklen rodit drugogo!

\cdash--* Gospozha Poklen nikogda bolee ne rodit takogo, i nikakaya drugaya gospozha v techenie neskolp1kih stoletii0 takogo ne rodit.

\cdash--* Vy menya izumlyaete, sudarp1!

\cdash--* YA i sam izumle0n. Poi0mite, chto po proshestvii tre0h vekov, v dale0koi0 strane, ya budu vspominatp1 o vas tolp1ko potomu, chto vy syna gospodina Poklena derzhali v rukah.

\cdash--* YA derzhala v rukah i bolee znatnyh mladencev.

\cdash--* CHto ponimaete vy pod slovom \cdash--- znatnyi0? E1tot mladenec stanet bolee izvesten, chem nyne carstvuyushchii0 korolp1 vash Lyudovik {\selectlanguage{english}XIII}, on stanet bolee znamenit, chem sleduyushchii0 korolp1, a e1togo korolya, sudarynya, nazovut Lyudovik Velikii0 ili korolp1-solnce! Dobraya gospozha, estp1 dale0kaya strana, vy ne znaete ee0, e1to \cdash--- Moskoviya. Naselena ona lyudp1mi, govoryashchimi na strannom dlya vashego uha yazyke. I v e1tu stranu vskore proniknut slova togo, kogo vy sei0chas prinimaete. Nekii0 polyak, shut carya Petra Pervogo, uzhe ne s vashego, a s nemeckogo yazyka perevede0t ih na varvarskii0 yazyk.

\ldots

\rightline{\small\slshape M.\,Bulgakov, ZHiznp1 \mbox{g-na} de~Molp1era, 1932--33\,gg.}

\newpage
\normalfont\parindent0pt\raggedbottom
Proverka ligatur shrifta:\medskip\nopagebreak\par
\tabcolsep3\tabcolsep
\long\def\ligtest
{\begin{tabular}{@{}*{3}{l}@{}}\hline\\[-2ex]
{\normalfont\selectlanguage{english}YO [E0]}:           \quad E0 \\
{\normalfont\selectlanguage{english}ZHE [Z1, ZH, Zh]}:  \quad Z1, ZH, Zh \\
{\normalfont\selectlanguage{english}I SHORT [I0]}:      \quad I0 \\
{\normalfont\selectlanguage{english}KHA [KH, Kh, H]}:   \quad KH, Kh, H \\
{\normalfont\selectlanguage{english}TSE [TS, Ts, C]}:   \quad TS, Ts, C \\
{\normalfont\selectlanguage{english}CHE [CH, Ch]}:      \quad CH, Ch \\
{\normalfont\selectlanguage{english}SHA [SH, Sh]}:      \quad SH, Sh \\
\multicolumn{2}{@{}l}{{\normalfont\selectlanguage{english}SHCHA [XQ, Xq, SHCH, Shch]}:\quad XQ, Xq, SHCH, Shch}\\
{\normalfont\selectlanguage{english}HARD SIGN [P2]}:    \quad P2 &
{\normalfont\selectlanguage{english}SOFT SIGN [P1]}:    \quad P1 \\
{\normalfont\selectlanguage{english}REV. E [E1]}:       \quad E1 \\
{\normalfont\selectlanguage{english}YU [J2, YU, Yu]}:   \quad J2, YU, Yu\\
{\normalfont\selectlanguage{english}YA [J1, YA, Ya]}:   \quad J1, YA, Ya \\
{\normalfont\selectlanguage{english}UKR. E [E2]}:       \quad E2 \\
{\normalfont\selectlanguage{english}DJE [D1, DJ, Dj]}:  \quad D1, DJ, Dj  &
{\normalfont\selectlanguage{english}DZHE [D2]}:         \quad D2   &
{\normalfont\selectlanguage{english}S [D3]}:            \quad D3\\
{\normalfont\selectlanguage{english}LJE [L1, LJ, Lj]}:  \quad L1, LJ, Lj \\
{\normalfont\selectlanguage{english}TSHE [C1]}:         \quad C1 \\
{\normalfont\selectlanguage{english}I [I1]}:            \quad I1 \\
{\normalfont\selectlanguage{english}NJE [N1, NJ, Nj]}:  \quad N1, NJ, Nj \\
{\normalfont\selectlanguage{english}NUMBER SIGN [N0]}:  \quad N0
\\[1ex]
{\normalfont\selectlanguage{english}yo [e0]}:       \quad e0 \\
{\normalfont\selectlanguage{english}zhe [z1, zh]}:  \quad z1, zh \\
{\normalfont\selectlanguage{english}i short [i0]}:  \quad i0 \\
{\normalfont\selectlanguage{english}kha [kh, h]}:   \quad kh, h \\
{\normalfont\selectlanguage{english}tse [ts, c]}:   \quad ts, c \\
{\normalfont\selectlanguage{english}che [ch]}:      \quad ch \\
{\normalfont\selectlanguage{english}sha [sh]}:      \quad sh \\
{\normalfont\selectlanguage{english}shcha [xq, shch]}:    \quad xq, shch \\
{\normalfont\selectlanguage{english}hard sign [p2]}:\quad p2 &
{\normalfont\selectlanguage{english}soft sign [p1]}:\quad p1 \\
{\normalfont\selectlanguage{english}rev. e [e1]}:   \quad e1 \\
{\normalfont\selectlanguage{english}yu [j2, yu]}:   \quad j2, yu \\
{\normalfont\selectlanguage{english}ya [j1, ya]}:   \quad j1, ya \\
{\normalfont\selectlanguage{english}ukr. e [e2]}:   \quad e2 \\
{\normalfont\selectlanguage{english}dje [d1, dj]}:  \quad d1, dj  &
{\normalfont\selectlanguage{english}dzhe [d2]}:     \quad d2  &
{\normalfont\selectlanguage{english}s [d3]}:        \quad d3\\
{\normalfont\selectlanguage{english}lje [l1, lj]}:  \quad l1, lj \\
{\normalfont\selectlanguage{english}tshe [c1]}:     \quad c1 \\
{\normalfont\selectlanguage{english}i [i1]}:        \quad i1 \\
{\normalfont\selectlanguage{english}nje [n1, nj]}:  \quad n1, nj \\
\hline
\end{tabular}\par}


{\rmfamily\ligtest}
{\itshape\ligtest}
{\ttfamily\ligtest}
{\sffamily\ligtest}


\end{document}
